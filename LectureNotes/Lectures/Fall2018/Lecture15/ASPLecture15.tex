% ==============================================================================



\documentclass[../../../Master/AppliedStochastics.tex]{subfiles}



% ==============================================================================


\author{Andrew}
\date{31 October 2018}


% ==============================================================================
%
\begin{document}
%
% ==============================================================================


\makelecture


\section*{10/31}


\section*{Cell Proliferation and Cancer}


The following setup is the motivation for a later theorem on Hitting 
Probabilities. Mutations can affect the rates of cell proliferation (splitting) 
and death. Suppose we counted the number of occurences of a particular mutation 
in each of many tissue samples. We wonder if the distribution of mutation 
numbers is consistent with a \underline{driver mutation}, i.e. one with a 
higher than normal rate of proliferation\footnote{Here, normal would mean that 
the rate of cell proliferation equals the rate of cell death.}.


\section*{Model}


Let $X_t$ be the number of mutated cells at time $t$, so $X_t \in 
\mathbb{N}_{\geq 0}$, and assume that $X_0 =1$ (so we start with one mutated 
cell). 

Assume that each cell divides at rate $\lambda$ and dies at rate $\mu$. This 
means that
\begin{align*}
	X_t &\to X_t + 1 \quad \text{at rate $\lambda X_t$}\\
	X_t & \to X_t - 1 \quad \text{at rate $\mu X_t$}.
\end{align*}

Questions:
\begin{enumerate}
	\item What is the probability that the mutation ``grows?" That is, for some 
	large $N$, what is 
	\[
		\bb P\{X_t = N \text{ before } X_t \text{ hits } 0 \}= ?
	\]
	\item If the mutation is not a driver but present due to an ongoing 
	mutation of rate $a$, what is the expected distribution of the number of 
	mutations per tissue sample?
\end{enumerate}


\section*{Hitting Probabilities}


To answer question 1 we need a bit more theory. Suppose $(X_t)_{t\geq 0}$ is a 
Continuous Time Markov Chain with generator $G$ on a state space $\mathcal{X}$. 
For any subset $A \subset \mathcal{X}$, we define
\[
	\tau_A = \inf\{t\geq 0 \mid X_t \in A\},
\]
which is the first time that the Markov chain lands in $A$. If $A=\{x\}$ we 
just write $\tau_x$, which is the first time the Markov chain hits $x$. 
We now state the theorem:


\begin{theorem}[Harmonic Functions]
Suppose that $A,B\subset X$ are disjoint subsets with the property that 
\[
	\bb p\{ \tau_{A\cup B} <\infty\} = 1.
\]
For $x\in \mathcal{X}$ we define 
\[
	h(x) = \bb p[x] \{\tau_A <\tau_B\} := \bb p\{\tau_A<\tau_B \mid X_0 = x\}.
\]
Then the function $h$ is the unique function satisfying the following two 
conditions:
\begin{enumerate}
	\item $
		h(x) = \begin{cases}
			1 & \text{ for } x \in A\\
			0 & \text{ for} x\in B
		\end{cases}
	$
	\item $Gh(x) = 0$ for $x\notin A\cup B$. 
\end{enumerate}
\end{theorem}

To prove the theorem we need two lemmas.
\begin{lemma}
Let $T_1,\dots, T_n$ be independent random variables with $T_i\sim 
\text{Exp}(\lambda_k)$. Let $T := \min\{T_1,\dots,T_n\}$. Then $T\sim 
\text{Exp}\left(\sum_{k=1}^n \lambda_k\right)$ and 
\[
	\bb p\{T_k = T\} = \frac{\lambda_k}{\lambda_1+\dots+\lambda_n}.
\]
\end{lemma}

\begin{proof}[Proof of Lemma 1]
	\begin{align*}
		\bb p\{T >t \} &= \bb p\{T_k > t \quad \forall k\}\\
		&= \prod_{k=1}^n \bb p\{T_k > t\} &&\text{(by independence)}\\
		&= \prod_{k=1}^n e^{-\lambda_k t} \\
		&= e^{-t \sum_{k}\lambda_k}
	\end{align*}
	which establishes the first claim. For the second claim, first note that
	\begin{align*}
		\bb p\{ T_k \in dt, T_j > t \quad \text{for } j\neq k\} &= \lambda_k 
		e^{-\lambda_k t} \prod_{j\neq k} e^{-\lambda_j t}\\
		&= \lambda_k e^{-t \sum_j \lambda_j},
	\end{align*}
	so we now integrate over $t$ to get
	\[
		\bb p\{T_k = T\} = \lambda_k \int e^{-t \sum_j \lambda_j}\,dt = 
		\frac{\lambda_k}{\lambda_1+\dots + \lambda_n}.
	\]
\end{proof}

%\begin{lemma} 
%Let $\tau_+ = \inf\{ t\geq 0 \mid X_t \neq X_0\}$ be the time of the first 
%jump. Then, if $X_{0} = x$, then $\tau_+ \sim \text{Exp}(-G_{xx})$.
%
%\end{lemma}



% ==============================================================================
%
\end{document}
%
% ==============================================================================
