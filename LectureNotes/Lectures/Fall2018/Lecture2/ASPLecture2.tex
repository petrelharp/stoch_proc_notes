% ==============================================================================



\documentclass[../../../Master/AppliedStochastics.tex]{subfiles}



% ==============================================================================


\author{Maya}
\date{26 September 2018}


% ==============================================================================
%
\begin{document}
%
% ==============================================================================


\makelecture


A note on the previous example:
    If we start with Brownian motion and incorporate exponential decay,
        a more general central limit theorem applies
    (for more information, look up Lindeberg-Feller condition).


\begin{definition}
    Random variables $Z_1, \dots, Z_n$
        (which we can think of as a random vector of length $n$,
            $(Z_1, \dots, Z_n)$)
    are \emph{jointly Gaussian} if for any $a_1, \dots, a_n \in \bbR$,
    we have
    \begin{equation*}
        \sum_{k = 1}^n a_k Z_k \sim N(m, \sigma^2)
    \end{equation*}
        for some $m$, $\sigma^2$.
    That is, these random variables are jointly Gaussian if and only if
        any linear combination of them is univariate Gaussian.
\end{definition}


\begin{definition}
Suppose that the random variables $(Z_1, \dots, Z_n)$ are jointly Gaussian.
Let $Z=(Z_1, \dots, Z_n)$.
Define a vector of means
\begin{equation*}
    \mu = \left(\begin{array}{ccc}
            \mu_1 \\ \vdots \\ \mu_n
          \end{array}
          \right)\,,
\end{equation*}
    where $\mu_i$ is the mean of $Z_i$.
Define the $n \times n$ covariance matrix
    $\Sigma = \left\{\Sigma_{i, j}\right\}$,
    where $\Sigma_{i, j} = \cov[Z_i, Z_j]$.
Then we write $Z \sim N(\mu, \Sigma)$.
\end{definition}


Now, suppose that we have $Z$, $\mu$, and $\Sigma$ as in the above definition,
    and let $a_1, \dots, a_n \in \bbR$.
Set
\begin{equation*}
    a = \left(\begin{array}{c}
            a_1\\ \vdots\\ a_n\\
        \end{array}\right)\,.
\end{equation*}
Then we know that $\displaystyle\sum_{k = 1}^n a_k Z_k \sim N(m, \sigma^2)$
    for some $m$, $\sigma^2$.
A small amount of algebraic manipulation yields the following equations:
\begin{equation*}
    m = \sum_{k = 1}^n a_k \mu_k = a^T \mu
\end{equation*}
    and
\begin{equation*}
    \sigma^2 = \sum_{1\leq i, j\leq n} a_i \Sigma_{i, j} a_j = a^T \Sigma a\,.
\end{equation*}


\begin{example}
We can have random variables that are marginally Gaussian
    (that is, each of them have Gaussian distributions themselves)
    but that are not jointly Gaussian.
Let $X \sim N(0, 1)$ and $Y \sim N(0, 1)$ be independent random variables.
Let $Z = \mathrm{sign}(X)|Y|$.
So $X$ and $Z$ are ``almost'' independent,
    but have the same sign as each other.
Then $X$ and $Z$ both have Gaussian distributions,
    but $X$ and $Z$ are not jointly Gaussian.
\end{example}


\begin{definition}
    A \emph{Gaussian process} on an index set $T$ is a collection of
        random variables $\{X_t\}_{t \in T}$ such that
            for any $n \in \bbN$, for any $(t_1, \dots, t_n) \in T^n$,
            $(X_{t_1}, \dots, X_{t_n})$ is jointly Gaussian.
    It is \emph{centered} if $\E[X_t] = 0$ for all $t$.
\end{definition}


\begin{example}
Let $Z \sim N(\mu, \Sigma)$ as above, % where $Z = (Z_1, \dots, Z_n)$;
    then it is a Gaussian process on $\{1, \dots, n\}$.
\end{example}


\begin{example}
Let $\{B_t\}_{t \geq 0}$ be a Brownian motion;
    then it is a Gaussian process on $[0,\infty)$.
\end{example}


\section[Facts about Gaussian processes]{%
    Facts about jointly Gaussian variables\\ and Gaussian processes}


\begin{enumerate}[(1)]
    \item
    If $Z \in \bbR[n]$ and $Z \sim N(\mu,\Sigma)$
        (that is, $Z$ is an $n$-dimensional multivariate Gaussian),
        then
    \begin{equation*}
        \E[f(Z)] = \int_{\bbR[n]}
            f(x) \dfrac{1}{\sqrt{(2\pi)^n \mathrm{det}(\Sigma)}}
            \exp\left(\dfrac{- (x - \mu)^T \Sigma^{-1} (x - \mu)}{2}\right)
                \dif x.
    \end{equation*}
    
    \item
    The distribution of a Gaussian process is determined by its mean, 
        $\mu(t)=\E[X_t]$ and covariance
            (also called covariance kernel) $\sigma^2(s, t) = \cov[X_s, X_t]$.
    
    \item
    Covariance kernels are positive semidefinite.
    Given a Gaussian process on $T$, where $T$ is a measure space,
        we can use this to define an inner product on some subset of
            the set of functions $T \to \bbR$
    (I don't know what the subset is that we need to take
        in order for this definition to make sense).
    Given $f, g \colon T \to \bbR$, define
    \begin{equation*}
        \inner{f, g}_{\sigma} \defeq \sum_{s \in T} \sum_{t \in T}
            f(s) g(t) \sigma^2(s, t)
    \end{equation*}
        and $\norm{f}_\sigma^2 = \inner{f, f}_\sigma \geq 0$.
    
    
    \textbf{Example}
    For Brownian motion,
        $\E[B_t]=0$, $\cov[B_s, B_t] = \sigma^2(s, t) = \mathrm{min}(s, t)$.
    So, for $f, g \colon [0, \infty) \to \bbR$,
    \begin{equation*}
        \inner{f, g}_\sigma = \int_0^\infty \int_0^\infty
            f(s) g(t) \mathrm{min}(s, t) \dif s \dif t\,.
    \end{equation*}


    \item
    If you have a linear space $V$,
        with a symmetric positive definite inner product $\inner{\cdot, \cdot}$,
        and a countable orthonormal basis $\{\phi_k\}_{k = 1}^{\infty}$,
            then you can define a centered isomorphic Gaussian process,
            ie. a Gaussian process on $V$ such that $\E[X_t] = 0$
                for all $t \in V$ and $\cov[X_s, X_t] = \inner{s, t}$.
    That is, you can construct random variables
        and index them using $V$ in such a way that
        the covariance of two random variables is equal to
        the inner product of their indices.


    This can be done in the following way.
    Let $\{Z_k\}_{k=1}^\infty$ be independent, identically distributed (iid)
        variables each with distribution $N(0, 1)$.
    For $t \in V$, which we can write as
        $t = \sum_{k = 1}^\infty \inner{t, \varphi_k} \varphi_k$,
        define $X_t = \sum_{k=1}^\infty \inner{t, \varphi_k} Z_k \in \bbR$.
    Because $Z_k$ are independent, and each have variance $1$,
    \begin{equation*}
        \cov[Z_i, Z_j] = \left\{\begin{array}{cc}
                            1 & \text{if } i = j \\
                            0 & \text{if } i \neq j \\
                         \end{array}\right.
    \end{equation*}
        and so
    \begin{align*}
        \cov[X_s, X_t] &= \cov\left[\sum_k \inner{s, \varphi_k} Z_k,
                            \sum_j \inner{t, \varphi_j} Z_j\right] \\
                       &= \sum_{j, k} \inner{s, \varphi_k} \inner{t, \varphi_j}
                            \cov[Z_k, Z_j] \\
                       &= \sum_j \inner{s, \varphi_j} \inner{t, \varphi_j} \\
                       &= \inner{s, t}\,.
    \end{align*}


    \begin{example}
    Suppose we wanted to construct a Gaussian process with covariance matrix
    \begin{equation*}
        \Sigma = \left(\begin{array}{ccc}
                1 & \frac{1}{2} & 0\\
                \frac{1}{2} & 1 & \frac{1}{2}\\
                0 & \frac{1}{2} & 1\\
                 \end{array}\right)\,.
    \end{equation*}
    
    
    That is, we want to construct random variables
        $X_{v_1}$, $X_{v_2}$, $X_{v_3}$, where $v_1, v_2, v_3 \in \bbR[3]$,
            such that for any pair $(X_{v_i}, X_{v_j})$, we have 
            $\cov[X_{v_i}, X_{v_j}] = \inner{v_i, v_j} = \Sigma_{i, j}$,
        where $\Sigma_{i, j}$ is the entry in the $i$th row and $j$th column
            of our covariance matrix.
    
    
    If we start with three independent random variables,
        each with distribution $N(0, 1)$,
        we can use the process above to construct $X_{v_1}, X_{v_2}, X_{v_3}$.
    In order to do this, however,
        we first need to determine what $v_1$, $v_2$, and $v_3$ are.
    To do this,
        we will use the fact that we need $\inner{v_i, v_j} = \Sigma_{i,j}$.
    
    
    (I wasn't able to write down anything coherent about the discussion
        about how to do this/why it exists, so this is a gap.)
    
    
    Once we have $v_1, v_2, v_3$,
        then we can start with $Z_{v_1}, Z_{v_2}, Z_{v_3}$,
        independent and each with distribution $N(0, 1)$,
        we can construct $X_{v_1}, X_{v_2}, X_{v_3}$
        using the process described above.
    \end{example}


    \item
    Linear transformations of Gaussians are Gaussian:
        if $Z = (Z_1, \dots, Z_n) \sim N(\mu, \Sigma)$
            and $A \in \R^{k \times n}$, then $A Z \sim N(A \mu, A \Sigma A^T)$.
    
    \begin{example}
        Let $\Sigma= k k^T$ be the Cholesky decomposition of $\Sigma$,
            and let $Z = (Z_1, \dots, Z_n)$ independent,
            each with distribution $N(0, 1)$.
    Then if we let $(X_1, X_2, X_3) = X = k Z$,
        then $X \sim N(0, k I k^T = \Sigma)$.
    \end{example}
    
    If $X, Y$ are jointly Gaussian,
        then $X$ and $Y$ are independent if and only if $\cov[X, Y] = 0$.
\end{enumerate}


% ==============================================================================
%
\end{document}
%
% ==============================================================================
