% ==============================================================================



\documentclass[../../../Master/AppliedStochastics.tex]{subfiles}



% ==============================================================================


\author{Martin}
\date{14 November 2018}


% ==============================================================================
%
\begin{document}
%
% ==============================================================================


\makelecture


\section{Levy Processes}

\begin{definition}
	A \underline{Levy process} with 
	drift rate $\alpha$,
	diffusion rate $\sigma,$
	and jump kernel $\nu$
	is a stochastic process with distribution $X_0 = 0$
	%
	\begin{equation}
	X_t
	=\alpha t 
	+ \sigma B_t 
	+ \int_{0}^{t} \int_{-\infty}^{\infty} x N(ds , dx) \label{Levy}
	\end{equation}
	%
	where $B_t$ is Brownian motion,
	so that $\E \left[ B_t \right] = 0$
	and $\var \left[ B_t \right] = t,$
	and where $N$ is a Poisson point process, independent from $B_t,$ on
	$\left [ 0 , \infty \right) \times \R$
	with intensity measure $dt \nu(dx).$ 
\end{definition}

\textbf{Fact:} (Levy-Khinchine)
	Any Markov process on $\R$ with $X_0 = 0$ and stationary independent 
	increments is Levy.
	
	i.e. a Markov process such that
	\begin{enumerate}
		\item the distribution of the increment $X_{t + h} - X_t$ depends only 
		on $h.$ 
		\item if $a < b \leq c < d$ then $(X_d - X_c)$ is independent of $(X_b 
		- X_a).$
	\end{enumerate} 

Note that $(\ref{Levy})$ makes sense if the jumps are absolutely summable, 
i.e. let $N = \sum_{i} \delta_{(t_i ,x_i)}.$
Then the "jump component" of $X$ is 
$$
\begin{aligned}
J_t 
&= \int_{0}^{t} \int _{-\infty}^{\infty} x N(ds, dx) \\
&= \sum_{i : t_i \leq t} x_i
\end{aligned}
$$
and we want $\sum |x_i| , \infty.$

Since jumps with $|x_i| > 1$ happen only finitely many times, absolute 
summability can be determined by looking at jumpes with $|x_i| < 1.$
Checking the expected value of this sum we have
$$
\begin{aligned}
\E \left[ \sum_{\stackrel{i : |x_i| < 1,}{ t_i < t}} |x_i|  \right] 
& = \E \left[ \int_{0}^{t} \int_{-1}^{1} |x| N(ds, dx)          \right] \\ 
& = t \int _{-1}^{1} |x| \nu dx
\end{aligned}
$$

which is finite if $\int _{-\infty}^{\infty} \min( |x|, 1 ) \nu(dx) < \infty.$

Properties of $X_t:$

\begin{enumerate}
	\item $X$ has generator
	\begin{equation}
	Gf(x)=\alpha f'(x) + \frac{\sigma^2}{2} f''(x) + \int_{-\infty}^{\infty} 
	\left( f(x+y) - f(x) \right) \nu(dy)
	\end{equation}
	
	\item (Levy-Khinchine formula)
	The characteristic function of $X_t$ is given by
	\begin{equation}
	\varphi(t, u)
	= \E \left[ e^{i u X_t} \right]
	= e^{t\psi(u)}
	\end{equation}
	where $\psi(u)=i \alpha u - \frac{\sigma^2}{2} u^2 + 
	\int_{-\infty}^{\infty} \left( e^{i u x} - 1 \right) \nu(dx)$
	since
	\begin{equation}
	\E \left[ e^{i u X_t} \right] 
	= \E \left[ e^{i u \alpha t} e^{i u \sigma B_t} e^{i u \int_{0}^{t} 
	\int_{-\infty}^{\infty} x N(ds, dx) } \right]
	\end{equation}
\end{enumerate}

\begin{example}Stable Subordinator of Index 1/2

Let $(B_t)_{t \geq 0}$ be Brownian motion and define
$\tau_r = \inf \left\lbrace t \geq 0 : B_t \geq r \right\rbrace.$
Then $(\tau_r)_{r \geq 0 }$ Levy with
$
\alpha=0,
\sigma=0,
\nu(dx)=x^{-3/2}dx.
$
\end{example}

\section{Levy Processes continued}

\begin{definition}
	A \underline{subordinator} is a nondecreasing Levy process.
\end{definition}

Let $(B_t)_{t \geq 0}$ be Brownian motion.
Set $t_x = \inf \left\lbrace t \geq 0 : B_t \geq x \right\rbrace.$ 
Then $(\tau_x)_{x \geq 0}$ is a nondecreasing Markov process with stationary, 
independent increments
and so is a subordinator.

For $x,y \geq 0,$ we have, setting $\tilde{B}_t = B_{\tau_x + t} - x,$
$$
\begin{aligned}
tau_{x + y} - \tau_x 
& = \inf \left\lbrace t \geq \tau_x : B_t = x + y \right\rbrace 
- \inf \left\lbrace t \geq 0 : B_t = x \right\rbrace \\
& = \inf \left\lbrace t \geq 0 : \tilde{B}_t = y \right\rbrace
- 0 \\
& \stackrel{d}{=} \tau_y
\end{aligned}
$$

which implies that $\tau$ has stationary increments.

Moreover, 
$(\tau_{x + y})_{y \geq 0}$
only depends on $(\tilde{B}_t)_{t \geq 0 }$
which is independent of $(B_s)_{0 \leq s \leq \tau_x}$
so $\tau$ is Markov.

\textbf{Scaling:}
Since $\overline{B}_s := \frac{1}{c} B_{c^2s})_{s \geq 0} 
\stackrel{d}{=}(B_s)_{s \geq 0}$
and 
$$
\begin{aligned}
\tau_x 
& = \inf \left\lbrace t \geq 0 : B_t \geq x \right\rbrace \\
& \stackrel{d}{=} \inf \left\lbrace t \geq 0 : \dfrac{1}{c} B_{c^2t} \geq x 
\right\rbrace  \\
& = \frac{1}{c^2} \inf \left\lbrace t \geq 0 : B_t \geq c x \right\rbrace \\
& = \frac{\tau_{c x}}{c^2}
\end{aligned}
$$

we have 
$\tau_x \stackrel{d}{=} x^2\tau_1.$

To find the distribution of $\tau$ we look at
\begin{equation}
\P \left\lbrace \tau_1 < t \right\rbrace 
= \P \left\lbrace \sup_{0 \leq s \leq t} B_s \geq 1 \right\rbrace
= \P \left\lbrace B_t \geq 1 \right\rbrace + \P \left\lbrace b_t < 1, M_t \geq 
1 \right\rbrace \label{tau}
\end{equation}
where 
$M_t 
= \sup \left\lbrace B_s : 0 \leq s \leq t \right\rbrace.$

\begin{lemma}Reflection Principle:
	
	$\P \left\lbrace B_t < 1, M_t \geq 1 \right\rbrace 
	= \P \left\lbrace B_t \geq 1 \right\rbrace$
	i.e. 
	$$
	\left( W_s = \left\lbrace\begin{array}{cc}
	B_s 		& \text{ if } s \leq \tau_1 \\
	1-(B_s-1) 	& \text{ if } s \geq \tau_1
	\end{array}
	\right.
	\right)
	\stackrel{d}{=} (B_s)_{ s \geq 0 }
	$$
\end{lemma}

Combining this Lemma with (\ref{tau}) we have
$$
\begin{aligned}
\P \left\lbrace \tau_1 < 1 \right\rbrace 
& = 2 \P \left\lbrace B_t \geq 1 \right\rbrace 					\\
& = 2 \P \left\lbrace B_1 \geq \frac{1}{\sqrt{t}} \right\rbrace	\\
& = \frac{2}{\sqrt{2\pi}}\int_{1/\sqrt{t}}^{\infty} e^{-x^2/2} dx 	\\
& = \frac{1}{\sqrt{2 \pi}} \int_{0}^{t} s^{-3/2} e^{-1/2s} ds
\end{aligned}
$$
where the last equality follows from setting $s=\frac{1}{x^2}.$
Thus,
$\P \left\lbrace \tau_x < t \right\rbrace 
= \P \left\lbrace x^2\tau_1 < t \right\rbrace 
= \P \left\lbrace \tau_1 < t/x^2 \right\rbrace 
$
so the density of $\tau_x$ is given by
$$
\frac{1}{\sqrt{2 \pi}}\frac{1}{x^2} \left(\frac{\tau}{x^2}\right)^{-3/2} 
e^{-x^2/2t} dt.
$$

Recall that $\tau$ is a Levy process.
$$
\begin{aligned}
\bb E \left[ e^{- \lambda \tau} \right] 
& = \int_{0}^{\infty} e^{- \lambda t} \frac{1}{\sqrt{2 \pi}} t^{-3/2} e^{-1/2t} 
dt \\	
& = e^{- \sqrt{2\lambda}} \\
& = \exp \left( \int_{0}^{\infty} (e^{-\lambda t } - 1) \frac{x^{-3/2}}{\sqrt{2 
\pi}} dx \right)
\end{aligned}
$$
and at the same time,
$$
\begin{aligned}
\bb E \left[ e^{-\lambda \tau} \right]
= \exp \left( \int_{0}^{\infty} (1 - e^{-\lambda t }) \nu(dx) \right)
\end{aligned}
$$

where $\nu$ is the jump measure for $\tau.$ It follows that 
$\nu(dx)=\frac{x^{-3/2}}{\sqrt{2 \pi}}dx.$
So,
$\tau_t = \int_{0}^{t} \int_{0}^{\infty} x N(dt, dx)$
where $N$ is a Poisson point process on $\left( 0 ,\infty \right)^2$ with 
intensity $\frac{x^{-3/2}}{\sqrt{2 \pi}}dtdx.$
Also,
$$
\bb E \left[ \tau_1 \right] 
= -\frac{d}{d\lambda} \bb E \left[ e^{-\lambda t_1} \right]\big|_{\lambda=0}
= \frac{1}{\sqrt{2 \pi}}e^{-\sqrt{2\lambda}}\big|_{\lambda=0}
= \infty
$$


% ==============================================================================
%
\end{document}
%
% ==============================================================================
