% ==============================================================================



\documentclass[../../../Master/AppliedStochastics.tex]{subfiles}



% ==============================================================================


\author{Isaac}
\date{21 November 2018}


% ==============================================================================
%
\begin{document}
%
% ==============================================================================


\makelecture


\subsection{Critical Branching}


\begin{definition}

Let $(V_k)_{k=1} ^\infty$ be defined by $(V_{k+1} | V_k) = \sum_{j=1} ^{V_k} 
M_{kj}$ for $V_k \in \bbZ$, where 
$$
\{ M_{kj} = \# \text{(offspring of indiv $j$ at time $k$)} \} 
$$
are iid with distribution 
$$
M = 
\begin{cases}
0 & \text{with probability $p$} \\
1 & 1 - 2p\\
2 & p
\end{cases}
$$
Notice the \emph{branching property}, that
$$
( (V_k)_{k=1} ^\infty | V_0 = n ) \disteq ( ( \underset{ \text{iid} }{ V_k 
^{(1)} + \cdots + V_k ^{(n)}} ) | V_0 ^{(1)} = \cdots = V_0 ^{(n)} = 1).
$$
We are interested in the total \# of ``individuals", that is, in 
$$
T = \sum_{k \geq 0} \sum_{j=1} ^{V_k} \mathbf{1}_{ \{ M_{kj} = 0 \} }
$$ 
[INSERT branching diagram]


\end{definition}


\begin{lemma}
$T < \infty$ almost surely.
\end{lemma}

More generally, for a branching process with offspring distribution $M$, what 
is the \emph{extinction probability}
$$
P_e = \P \{ \lim_{k \to \infty} V_k = 0 | V_0 = 1 \}
$$
that is that $V$ dies out. We can introduce some recursion here, as $V$ dies 
out iff the families of all first-generation offspring die out. Then, by the 
branching property, 
$$
P_e = \E [P_e^M] =: \varphi ( p_e)
$$
That is, define 
$$
\varphi (u) = \E [u^M] = \sum_{n \geq 0} u^n \P \{ M=n \},
$$
a concave function as $ \P \{ M=n \} \geq 0 \implies \varphi '' (u) \geq 0$ for 
$u \in [0,1]$. $P_e$ is a fixed point of $\varphi(u)$, and we can compute 
$$
\varphi (0) = \P \{ M = 0 \}, \quad \varphi (1) = 1,
$$
and $\varphi' (1) = \E [ M ] = \mu$. Assume this is finite, and recalling the 
concavity condition on $\varphi$, this leads to classifying conditions:

\begin{enumerate}
\item $\mu < 1$ (subcritical): $P_e = 1$ is the only solution.
\item $\mu = 1$ critical: $P_e = 1$ is the only solution.
\item $\mu > 1$ (supercritical): If $\mu > 1$ then $P_e < 1$. 
\end{enumerate}

[INSERT criticality plots]

\bigskip{}
\noindent{}Let 
$$
\psi (u) = \E [u^T | V_0 = 1 ] = \sum_{n \geq 0} u^n \P \{ T = n | V_0 = 1 \}.
$$ 
and
$$
u^T \disteq 
\begin{cases}
u &  M_p = 0 \\
u^T	&	M_p = 1\\
u^{T^{(1)}} \times u^{T^{(2)}} = (u^T)^2	&	M_p = 2
\end{cases}
$$
where we have an assumption that $T^{(1)}, T^{(2)}$ are iid $\sim T$. Then, 
since $\E [ u^{T^{(1)}} u^{T^{(2)}} ] = \E [ u^{T^{(1)}} ] \E 
[u^{T^{(2)}} ]  = \psi(u)^2$, 
$$
\psi(u) = pu + (1-2p) \psi(u) + p \psi(u)^2
$$
that is, solving the quadratic,
\begin{align*}
\psi(u) 
	&= 1 - \sqrt{1 - u} \\
	&= \sum_{n \geq 1} \frac{1}{2 \sqrt{\pi}} \frac{ \Gamma (n - \tfrac{1}{2}) 
	}{n!} u^n \text{ by Binomial Series} \\
	&\implies \P \{ T = n | V_0 = 1 \} = \frac{1}{ \sqrt{2 \pi} } \frac{ \Gamma 
	(n - \tfrac{1}{2}) }{n!}
\end{align*}

Stirling's formula gives the approximation $
\log \Gamma(z) 
	= z \log z - z + O(\log z)
$
and
\begin{align*}
\log \frac{ \Gamma(n - \tfrac{1}{2}) }{ \Gamma(n+1) }
	&= (n-\tfrac{1}{2}) \log (n - \tfrac{1}{2}) - (n - \tfrac{1}{2}) - (n+1) 
	\log (n+1) + (n+1) + O(\log n) \\
	&\approx \tfrac{3}{2} \log (n)
\end{align*}
giving the approximation 
$$
\P \{ T = n | V_0 = 1 \} \approx  \frac{1}{\sqrt{2} \pi} n^{-3/2}.
$$ 
That is, $\P \{ T > n \} \approx n^{-1/2}$. \newline{}

\noindent{}Let $H_n = T^{(1)} + \cdots + T^{(n)}$. For instance, if $T_n$ is 
the number of berries on a fully grown branch of a bush, then $H_n$ is the 
total number of berries after $n$ time steps (fully grown branch 
``generations"). What will make this stabilize? $n^{3/2}$, $n^{5/2}$, or 
$n^{1/2}$? Well, using the stability property we can see that $\frac{H_n}{n^2} 
\to \text{Stable}(\tfrac{1}{2})$.

\bigskip{}

Reminder: Stability Property says that if $Y_i$ are iid, $\P \{ Y_i > y \} \sim 
y^{-p}$ as $y \to \infty$, then $\frac{\Sigma Y_i}{n^{1/p}} \to$ Stable process 
of order ($p$). 

\bigskip{}

Then, $\left(X_t = H_{n_t} / n^2 \right)_{t \geq 0} \to (\tau)t )_{t \geq 0}$, 
BM hit time, which is a stable($1/2$) subordinator. That is, there will 
probably be a bush with $\approx n^2$ berries (after $n$ ``generations"):
$$
\E [ \# k : T^{(k)} > x] = n \P \{ T > x \} \sim n x^{-1/2} = 1
$$
if $x \propto n^2$, so $\P \{ T > n^2 \} \sim n^{-1}$.


% ==============================================================================
%
\end{document}
%
% ==============================================================================
