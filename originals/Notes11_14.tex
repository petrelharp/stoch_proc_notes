\documentclass[12pt]{article}
\usepackage[left=3cm,right=3cm,top=3cm,bottom=2cm]{geometry}

%%
% Add your macros here; they'll be included in pdf and html output.
%%

\newcommand{\Z}{\mathbb{Z}}    % integers
\newcommand{\N}{\mathbb{N}}    % naturals
\newcommand{\R}{\mathbb{R}}    % reals

\newcommand{\E}{\mathbb{E}}    % expectation
\renewcommand{\P}{\mathbb{P}}  % probability
\DeclareMathOperator{\sd}{sd}
\DeclareMathOperator{\var}{var}
\DeclareMathOperator{\cov}{cov}

\newcommand{\given}{\;\vert\;}

\usepackage{tikz}
\newtheorem{fact}[thm]{Fact}

\begin{document}
%--------------------------------------------------

\section{L\'evy  Processes}

\begin{defn}
A L\'evy  process  with drift rate $\alpha$, diffusion rate $\sigma$, and jump kernel $\nu$ is a stochastic process with distribution: 
\begin{itemize}
\item[\ ] $X_0 = 0$,
\item[$(*)$] $X_t = \alpha t + B_t + \int_0^t \int_{-\infty}^\infty xN(dx, dt)$
\end{itemize}
where $B_t$ is Brownian motion and $N \sim PPP$ on $[0, \infty) \times \R$ with intensity measure $dt \,\nu(dx)$.
\end{defn}

\begin{fact}
(L\'evy-Khinchine)
Any Markov process on $\R$ with $X_0 = 0$ and stationary independent increments is L\'evy. \\

\noindent i.e. a markov process with 
\begin{itemize}
\item[(a)] distribution of increment $X_{t+h} - X_t$ only depends on $h$
\item[(b)] If $a < b \leq c < d$,
$(X_d - X_c)$ independent of $(X_b - X_a)$ with 
$$\begin{aligned}
\int_{-\infty}^\infty \min(|x|, 1) \nu(dx) < \infty
\end{aligned}$$
\big( and $\nu([1,\infty)) + \nu((-\infty,1]) < \infty$ \big).
\end{itemize}
\end{fact}

\noindent
Note: $(*)$ makes sense if jumps are absolutely summable. 
i.e. let $N = \sum_i \delta_{(t_i, x_i)}$. Then the ``jump component" of $X$ is 
$$\begin{aligned}
J_t = \int_0^t \int_{-\infty}^\infty xN(dt,dx) = \sum_{i, t_i \leq t} x_i \qquad \text{where} \quad \sum_i |x_i| < \infty
\end{aligned}$$

\noindent Is $\sum_{|x_i| < 1,\ t_i < t} |x_i| < \infty\,?$
$$\begin{aligned}
\E\left[ \sum_{|x_i| < 1,\ t_i < t} |x_i| \right] 
	&= \E \left[ \int_0^t \int_{-1}^1 |x|N(dt,dx) \right] \\
	&= t \cdot\!\! \int_{-1}^1 |x| \nu\, dx < \infty.
\end{aligned}$$

\noindent \textbf{OR:} If you have faster accumulation of jumps near zero,
\begin{itemize}
\item[$(**)$] $\displaystyle X_t = \alpha t + \sigma B_t + 
	\int_0^t \int_{-1}^1 x \big[ N(ds,dx) - dx \,\nu(dx) \big] +
	\int_0^t \int_{|x| > 1} x N(ds,dx)$
\item[\ ] Need only $\displaystyle \int_{-\infty}^\infty \min(|x|^2, 1)\, \nu(dx) < \infty$ \\
\end{itemize}

\pagebreak
\noindent \textbf{Properties:}
\begin{itemize}
\item[(1)] Stationary independent increments.
\item[(2)] $X$ has generators
	$$\begin{aligned}
	Gf(x) = xf'(x) + \frac{\sigma^2}{2} f''(x) + \int_{-\infty}^\infty \big[f(x+y) - f(x) \big] \nu(dy)
	\end{aligned}$$
\item[(3)] L\'evy - Keinchine fomula:
$$\begin{aligned}
\E\big[ e^{i u X_t} \big] &= e^{t \Psi(u)} \\
\Psi(u) &= e^{i \alpha u} - \frac{\sigma^2}{2}u^2 + \int_{-\infty}^\infty \big[ e^{i u x} - 1 \big] \nu(dx) \\
e^{t\Psi(u)} &= \int_{-\infty}^\infty e^{i u x}\, \P^0 \{X_t = x\} \,dx
\end{aligned}$$
\end{itemize}

\begin{exmp} ``Stable Subordinators" \\

$(B_t)_{t \geq 0}$ \qquad \qquad $\tau_r = \inf\{t \geq 0 : B_t \geq r \}$ \\ \\

\pgfmathsetseed{26}
\begin{tikzpicture}[scale=2]
\draw[->] (0,-1) -- (0,2.5);
\draw[->] (-.2,0) -- (4.2,0);
\draw[dashed] (1.4,0) -- (1.4,.98) -- (0,.98) node[left] {$r_1$};
\draw[dashed] (2.405,0) -- (2.405,1.8) -- (0,1.8) node[left] {$r_2$};
\node at (1.4,-.13) {$\tau_{_1}$};
\node at (2.41,-.13) {$\tau_2$};
\draw (0,0)
\foreach \x in {1,...,100} {-- ++(.04,rand*.2)};
\end{tikzpicture}

\noindent \textbf{Claim:} $(\tau_r)_{r \geq 0}$ is L\'evy with $\alpha = 0$, $\sigma = 0$, and $N(dx) = x^{-3/2}dx$

\pgfmathsetseed{26}
\begin{tikzpicture}[scale=2]
\draw[->] (0,-1) -- (0,2.5);
\draw[->] (-.2,0) -- (4.2,0);
\node at (2,2.2) {\Large{$\tau_r$}};
\node at (4.4,0) {\large{$r$}};
\draw (0,0)
\foreach \x in {1,...,40} {-- ++(.1,abs{rand*.13})};
\end{tikzpicture} \\

``Subordinators" $\sim$ L\'evy process $X_i \nearrow \beta$, $t \nearrow \infty$ with
independent stationary increments. \\
\end{exmp}


\section{First Hitting Times and Brownian Motion}

Let $(B_t)_{t \geq 0}$ be Brownian motion, 
$$\begin{aligned}
\tau_x = \inf \big\{ t \geq 0 : B_t \geq x \big\}, \quad \text{and} \quad
M_t = \sup \big\{ B_s : 0 \leq s \leq t \big\}.
\end{aligned}$$
Then $(\tau_x)_{x \geq 0}$ is a Markov process with stationary independent increments, 
i.e. a non decreasing L\'evy process, aka \underline{subordinator}. \\

Why is this Markov?

Define $\tilde B_t$ by $\big( \tilde B_t = B_{\tau_{x}+t} - x \big)_{t \geq 0} := \big( B_t \big)_{t \geq 0}$. 
Then
$$\begin{aligned}
\tau_{x+y} - \tau_x 
	&= \inf \big\{ t \geq \tau_x : B_t = x+y \big\} \\
	&= \inf \big\{ t \geq 0 : \tilde B_t = y \big\} \\
	&= \tau_y,
\end{aligned}$$
which implies $\tau$ has stationary increments.

\subsection{Properties of This Process}

\textbf{Recall:} Brownian Scaling: 
$\big( \overline B_s  = \frac{\,1\,}{c} \!\cdot\! B_{c^2 s} \big)_{s \geq 0} = \big( B_s \big)_{s \geq 0}$. Then
$$\begin{aligned}
\tau_x
	&= \inf \big\{ t \geq 0 : B_t \geq x \big\} \\
	&= \inf \left\{ t \geq 0 : \,\frac{\,1\,}{c} \!\cdot\! B_{c^2 t} \geq x \right\} \\
	&= \frac{\,1\,}{c^2} \!\cdot \inf \left\{ t \geq 0 : \,B_{t} \geq cx \right\} \\[2mm]
	&= \frac{\,1\,}{c^2} \!\cdot \tau_{cx}.
\end{aligned}$$

So $\tau_x = x^2 \tau_{_1}$.

\begin{lemma} 
Reflection Principle:
$$\begin{aligned}
\P \{ B_t < 1, M_t \geq 1 \} = \P \{ B_t \geq 1 \}.
\end{aligned}$$


\end{lemma}
\begin{proof}
\noindent
There is a bijection obtained by reflecting the $(t \geq \tau_{_1})$-portion of the process 
across the line $x = 1$: \\

\pgfmathsetseed{4}
\begin{tikzpicture}[scale=1.5]
\draw[->] (0,-1) -- (0,2.5);
\draw[->] (-.2,0) -- (4.2,0);
\draw[dashed] (4,1.72) -- (0,1.72) node[left] {$1$};
\draw (0,0)
\foreach \x in {1,...,23} {-- ++(.04,-rand*.2)};
\draw[red] (.92,1.72)
\foreach \x in {24,...,100} {-- ++(.04,rand*.2)};
\end{tikzpicture}
\qquad
\pgfmathsetseed{4}
\begin{tikzpicture}[scale=1.5]
\draw[->] (0,-1) -- (0,2.5);
\draw[->] (-.2,0) -- (4.2,0);
\draw[dashed] (4,1.72) -- (0,1.72) node[left] {$1$};
\draw (0,0)
\foreach \x in {1,...,23} {-- ++(.04,-rand*.2)};
\draw[red] (.92,1.72)
\foreach \x in {24,...,100} {-- ++(.04,-rand*.2)};
\end{tikzpicture}

So $\left( W_s = \begin{cases} \hfill B_s \hfill & s \leq \tau_{_1} \\ 2 - B_s & s \geq \tau_{_1} \end{cases} \ \right)
= \big( B_s \big)_{s \geq 0}$. \\
\end{proof}

$\newline$
$$\begin{aligned}
\P\big\{ \tau_{_1} < t \big\} 
	&= 2\P\big\{ B_t > 1 \big\} \\[1mm]
	&= \frac{2}{\sqrt{2\pi\,}t} \int_1^\infty \exp \left( \frac{-x^2}{2t} \right) dx \\[2mm]
	&= 2\P\left\{ B_1 > \frac{1}{\sqrt{t\,}} \right\} \\[1mm]
	&= \frac{2}{\sqrt{2\pi\,}} \int_{\frac{1}{\sqrt{t\,}}}^\infty \exp \left( \frac{-x^2}{2} \right) dx \\[2mm]
	&= \sqrt{\frac{2}{\pi\,}} \int_0^t s^{-3/2} \cdot e^{-1/(2s)} ds 
		\hspace{30mm} \mbox{$\left(\text{letting } s = \frac{1}{\,x^2}\right)$} \\
\end{aligned}$$

{\renewcommand{\arraystretch}{2.5}
\begin{tabular}{|c|}
\hline
$\P\big\{ \tau_{_1} \in ds \big\} = \sqrt{\dfrac{2}{\pi\,}} \cdot s^{-3/2} \cdot e^{-1/(2s)} ds$ \\[2mm] \hline
\end{tabular}}
\ $\leftarrow$ density \\ 

$$\begin{aligned} 
\text{Now} \hspace{30mm} \\
\P\big\{ \tau_x < t \big\} 
	&= \P\big\{ x^2\tau_{_1} < t \big\} \\
	&= \P\left\{ \tau_{_1} < \dfrac{t}{\,x^2} \right\}. \\ \\ 
\text{So} \hspace{30mm} \\
\P\big\{ \tau_x \in dt \big\} 
	&= \dfrac{d}{dt} \Big(\quad ,\quad \Big) \\
	&= \frac{1}{\,x^2} \cdot \dfrac{d}{dt}\  \P\left\{ \tau_{_1} < \dfrac{t}{\,x^2} \right\} \\
	&= \sqrt{\frac{2}{\pi\,}} \cdot \frac{1}{\,x^2}\cdot \left( \frac{t}{\,x^2} \right)^{-\frac{3}{2}} \!\!\cdot 
			\exp \left( \frac{-x^2}{2t} \right). \hspace{40mm}
\end{aligned}$$

\noindent 
Note: must be a \underline{jump} L\'evy process (can't have drift component nor Brownian component).


\end{document}